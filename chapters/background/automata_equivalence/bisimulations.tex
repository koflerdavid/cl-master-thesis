\begin{definition}
    A category is a class of \textit{objects}, together with \textit{morphisms} between them.
    A morphism maps an \textit{origin} object to a \textit{destination} object.
    For every object, we require the existence of an \textit{identity} morphism $id$ from the object to itself.

    Also, we require composability between morphisms:
    Two morphisms can be \textit{composed} if the source object of the first is
    the destination object of the second.
    For two morphisms $f : B \rightarrow C$ and $g : A \rightarrow B$,
    this is described as $f \circ g : A \rightarrow C$.

    The identity morphism must behave as a neutral element regarding composition, i.e.,
    the composition of any morphism of a given category with an identity morphism
    must behave as the original morphism.
\end{definition}

\begin{example}
    The class of all possible sets as objects, together with functions between them as morphisms,
    yields the category \textbf{Set}.
    The identity function $id_A$ acts as the identity morphism for every set $A$.
    Clearly, function composition satisfies the above requirements.
\end{example}

\begin{example}
    Every graph $G$ with the set $V$ as vertices and the set $E$ as edges
    can be turned into a category with the set $V$ as objects
    using the \textit{Free Category} construction.
    To fulfil the category laws, its morphisms are drawn from
    the smallest superset $E'$ of $E$ fulfilling the following conditions:

    \begin{itemize}
        \item There shall be an edge from every vertex $v \in V$ to itself.
        \item For every pair of edges $e, f \in E$ and vertices $u, v, w \in V$,
              where $e$ leads from $v$ to $w$ and $f$ from $u$ to $v$,
              there shall be an edge $e \circ f$ from $u$ to $w$.
    \end{itemize}
\end{example}

\begin{definition}
    The morphisms of the category of \textit{small categories} are called \textit{functors}.
    A \textit{functor} $F : A \rightarrow B$ maps objects and mappings between categories.
    Applying a functor to a mapping is also called \textit{lifting}.
    Functors have to preserve the following properties:

    \begin{itemize}
        \item For every morphism $f$ and $g$ from $A$, $F (g \circ f)$
              must be equivalent to $F g \circ F f$.
        \item The identity morphism of an object $a$ from $A$, $F id_a $,
              must be lifted to $id_{F_a}$, the identity morphism of $F a$.
    \end{itemize}

    Functors that map a category to itself are called \textit{endofunctors}.
\end{definition}

\begin{definition}
    Given a category $\Cat{C}$ and an endofunctor $F : \Cat{C} \rightarrow \Cat{C}$,
    every object $A$ from $Cat{C}$, together with a morphism
    $\alpha : F(\Cat{C}) \rightarrow \Cat{C}$
    specifies an $F$-algebra with the object $A$ as \textit{carrier}.
\end{definition}

Algebras are mathematical objects that are equipped with some \textit{constructors}.
A constructor is a function that makes it possible to generate elements of
the carrier algebra\footnote{In the abstract definition, these are combined into a single
constructor by using the disjoint union of functions}, often from existing ones.

\begin{example}
    In the category \textbf{Set}, the function
    $[\textit{nil}, \textit{cons}] : \textbf{1} \sqcup X \times List(X) \rightarrow List(X)$
    with the carrier $List(X)$ is the algebra of lists of elements of the set $X$.
    This algebra has one constructor that maps the singleton set to the empty list,
    or prepends an element of $X$ to another list.
\end{example}

Coalgebras are mathematical structures which are not defined in terms of constructors
out of which objects are composed, but by destructors by which an object is "taken apart".
They are especially suited for infinite or circular data structures,
such as streams, graphs, automata, processes, etc.
Said differently, algebras \textit{construct} objects
while coalgebras \textit{observe} objects~\cite{Jacobs97atutorial}.

\begin{definition}
    Given a category $\Cat{C}$ and an endofunctor $F : \Cat{C} \rightarrow \Cat{C}$,
    every object $X$ from $\Cat{C}$, together with a morphism
    $\alpha : \mathcal{C} \rightarrow F(\mathcal{C})$
    specifies an $F$-coalgebra with the object $X$ as \textit{carrier}.
\end{definition}

\begin{example}
    Streams are an infinite data structure and therefore
    a prime example for a coinductive data structure.
    The function
    $\textit{next} : \textbf{Stream}(A) \rightarrow A \times \textbf{Stream}(A)$
    is a destructor that decomposes a stream of elements from $A$ into a head and a tail.
    The destructor \textit{next} can be used to define the functions \textit{head} and \textit{tail}.
    This way, a coalgebra might be viewed as having more than one destructor.
\end{example}

\begin{example}
    States of an NFA $M$ can be represented as a coinductive algebra as well.
    To make presentation easier, the destructor has been split into two functions:

    \begin{align}
        o: Q_M \rightarrow \{0, 1\}\quad &o(q) =
            \begin{cases}
                0 & \textbf{if } q \not\in F_M\\
                1 & \textbf{if } q \in F_M
            \end{cases}\\
        t: Q_M \rightarrow {\Powerset{Q_N}} ^{\Sigma_M} \quad & t(q)(a) = \delta_M(q, a)
    \end{align}
\end{example}

The usage of destructors instead of constructors requires the use of different proof principles.
Bisimulations are such a coinductive proof principle that is used to establish
a notion of \textit{observational equivalence}.
It compares the behavior of elements of the coinductive algebra,
i.e., if the destructors yield the same results for both elements, then these elements are concluded
to be observationally equivalent.
In the following, the application of this proof principle to automata is demonstrated.

\begin{definition}
    The greatest bisimulation over the states is the bisimilarity relation $\sim$.
\end{definition}

\begin{theorem}
    The bisimilarity relation $\sim$ is an equivalence relation.
\end{theorem}

\begin{proof}
    We have to prove the three properties of an equivalence relation:
    reflexivity, symmetry, and transitivity.

    Reflexivity is easy since obviously for all $s \in Q: s \in F \iff s \in F$
    Similarly, symmetry of $\iff$ gives us symmetry.
    Transitivity
\end{proof}

%\begin{definition}
%    A relation $R$ between the states of the DFA $M$ is a bisimulation up to $f$
%    if $f$ is a monotone function and if $R$ satisfies the following properties:
%
%    \begin{align}
%        \forall (p, q) \in R&: p \in F_M \iff q \in F_M \\
%        \forall (p, q) \in R, a \in \Sigma_M&: (\delta_M(p, a), \delta_M(q, a)) \in f(R)
%    \end{align}
%
%    Monotonicity of $f$ is defined as follows:
%    \begin{align}
%        \forall S \in \Powerset{Q \times Q}: S \subseteq f(S)
%    \end{align}
%
%    In the following, $(p, q) \in R$ will be written more succinctly as $p R q$.
%\end{definition}
%

%This equivalence relation is based on observable equality.
%Two states are equivalent under this equivalence relation iff they have the save
%behaviour, i.e., they accept the same language.
