The following naïve version of Hopcroft and Karp's algorithm
takes as input a DFA and a pair of states (usually the starting states)
and tries to compute a bisimulation relating them.
If it succeeds then the two states are proven to accept the same language.
If not, then the algorithm terminates and can be made to return a pair
of states that do not accept the same language.

\begin{definition}
    \begin{align*}
        \underline{\mathit{Naive(p_1, p_2)}}: &\quad (Q \times Q) \to \{\mathbf{true}, \mathbf{false}\} \\
        \text{(1) } & R := \emptyset, \mathit{todo} := \{(p_1, p_2)\} \\
        \text{(2) } & \text{while } \mathit{todo} \neq \emptyset \text{ do}\\
        & \text{(2.1)}\quad \mathit{todo} := \mathit{todo} \setminus \{(p', q')\}\\
        & \textbf{(2.2)}\quad \textbf{if } \mathbf{(p', q') \in} \textbf{ R} \textbf{ then goto (2)}\\
        & \text{(2.3)}\quad \text{if } o(p') \neq o(q') \text{ then return } \mathbf{false}\\
        & \text{(2.4)}\quad \mathit{todo} := \mathit{todo} \cup \bigcup_{a \in \Sigma}{(t_a(p'), t_a(q'))}\\
        & \text{(2.5)}\quad R := R \cup \{(p', q')\} \\
        \text{(3) } & \text{return } \mathbf{true}\\
    \end{align*}
\end{definition}

A discussion of this algorithm is in order.
It has to check whether $p_1$ and $p_2$ are equivalent,
and it does so by starting from an empty bisimulation,
which represents that nothing is known yet.
Since this is a coinductive proof principle, equivalence has to be determined according
to the data returned by the destructors, i.e., whether the states are both accepting or not,
and whether their state transition functions are equivalent.
The former is easy to check, the latter involves checking the equivalence of functions.
Fortunately, functions with finite domain can be compared by checking whether each possible
input leads to the same output.
In the present case, this leads to a pair of states for every symbol from the alphabet that has to be compared.
This is represented by a ``to-do''-list of constraints that has to be checked.
If the algorithm succeeds to empty this ``to-do''-list, then we have proven the existence of the initial constraint.

For every constraint, the first step is to check whether the bisimulation already contains it
because only then there is further work to do.
If the two states are both accepting, or both not accepting, then
the equivalence depends on the equivalence of the states that can be reached from the present states.
Therefore, for every input symbol of the automaton, the pair of states reachable using that symbol
is to be added to the ``to-do'' list.
Also, the constraint is removed from the ``to-do'' list and added to the bisimulation.

By adding a constraint to the bisimulation, the algorithm risks inserting wrong facts,
but these will eventually be discovered by processing the ``to-do'' list.
This manifests by detecting a constraint where one of the states is accepting and the other is not.
Similar to a broken zipper on a jacket, up to this point the algorithm succeeded in
relating states of the automaton, until a divergence is encountered.
On the other hand, if the algorithm manages to turn all constraints in the ``to-do'' list
into members of the bisimulation, then equivalence has been proven.

\todo{Insert example}

\todo{Insert proof}
