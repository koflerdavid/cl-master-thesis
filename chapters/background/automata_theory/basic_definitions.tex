\begin{definition}
    The set of all finite sequences of an alphabet $\Sigma$ is written as $\Sigma^\ast$.
    It is formally defined as
    \begin{align}
        \Sigma^\ast = \bigcup\limits_{n=0}^\infty \{c_1 c_2 \ldots c_n \mid \forall 1 \le i \le n: c_i \in \Sigma \}
    \end{align}
    Subsets of $\Sigma^\ast$ are often called languages.
\end{definition}

The empty sequence (containing no symbols at a all) is denoted by $\epsilon$.

\begin{theorem}
    Languages form a Kleene algebra with the following definitions:
    \begin{itemize}
        \item The empty set $\{\}$ for $0$
        \item The language $\{ \epsilon \}$ for $1$
        \item Set union for $+$
        \item Language concatenation for $\times$, which is defined as follows:
            \begin{align}
                L_1 L_2 \coloneqq \{ v w \mid v \in L_1, w \in w \}
            \end{align}
        \item Asterate for languages is defined as follows:
            \begin{align}
                L^\ast \coloneqq \bigcup\limits_{i = 0}^\infty L^i
            \end{align}
    \end{itemize}
\end{theorem}
