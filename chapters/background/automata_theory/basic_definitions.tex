\begin{definition}
    The set of all finite sequences of an alphabet $\Sigma$ is written as $\Sigma^\ast$.
    It is formally defined as
    \begin{align}
        \Sigma^\ast = \bigcup\limits_{n=0}^\infty \{c_1 c_2 \ldots c_n \mid c_i \in \Sigma \text{ for all } 1 \le i \le n \}
    \end{align}
    Subsets of $\Sigma^\ast$ are often called languages.
\end{definition}

The empty sequence (containing no symbols at all) is denoted by $\epsilon$.

\begin{definition}
    A \textit{Kleene algebra} is a set $A$ together with:
    \begin{itemize}
        \item Two designated elements $0$ and $1$
        \item Two binary operations $+, \cdot : A \rightarrow A$.
        \item A unary operation $(-)^\ast : A \rightarrow A$, also called \textit{asterate} or \textit{Kleene star}.
    \end{itemize}

    For all $a, b, c \in A$, the following axioms must hold:

    \begin{itemize}
        \item The operations $+$ and $\cdot$ must be associative:
            \begin{align}
                a \cdot (b \cdot c) &= (a \cdot b) \cdot c\\
                a + (b + c) &= (a + b) + c
            \end{align}

        \item $+$ must be commutative, idempotent, and distributive over $\cdot$:
            \begin{align}
                a + b &= b + a\\
                a + a &= a\\
                (a + b) \cdot c &= a \cdot c + b \cdot c
            \end{align}

        \item $0$ and $1$ are neutral elements for $+$ and $\cdot$, respectively;
            $0$ must be the annihilating element of $\cdot$:
            \begin{align}
                a + 0 &= a\\
                a \cdot 1 &= 1 \cdot a = a\\
                a \cdot 0 &= 0 \cdot a = 0
            \end{align}

        \item Given the partial order defined by $a + b = b \Rightarrow a \leq b$,
            the operation $\ast$ must behave as follows:
            \begin{align}
                1 + a \cdot a^\ast = 1 + a^\ast = a^\ast\\
                b + a \cdot c \leq c \Rightarrow a^\ast \cdot b \leq c\\
                b + c \cdot a \leq c \Rightarrow b \cdot a^\ast \leq c
            \end{align}

    \end{itemize}

\end{definition}

\begin{theorem}
    The simplest example are Boolean Algebras with $\vee$ for $+$, $\wedge$ for $\cdot$,
    and with the constant function $\mathbf{1}$ for $\ast$.
\end{theorem}

\begin{theorem}
    Languages form a Kleene algebra with the following definitions:
    \begin{itemize}
        \item The empty set $\emptyset$ for $0$.
        \item The language $\{ \epsilon \}$ for $1$.
        \item Set union for $+$.
        \item Language concatenation for $\times$, which is defined as follows:
            \begin{align}
                L_1 L_2 \coloneqq \{ v w \mid v \in L_1, w \in L_2 \}
            \end{align}
        \item Asterate for languages is defined as follows:
            \begin{align}
                L^0 &\coloneqq \{\epsilon\} \\
                L^n &\coloneqq L L^{n-1} \\
                L^\ast &\coloneqq \bigcup\limits_{i = 0}^\infty L^i,
            \end{align}
    \end{itemize}
\end{theorem}
