Taking the Brzozowski derivative of regular expressions is an interesting operation which, among others, can be used
to implement regular expressions in software, and also to convert regular expressions into finite automata.
It can also be extended to other classes of formal languages~\cite{parsing-with-derivatives}.
The operation has similarities to taking the (partial) derivative of arithmetic expressions.

\begin{definition}
    The derivative of a regular expression $r$ w.r.t. a letter $c$ is
    the set of all words in $\Lang{r}$ which start with $c$:

    \begin{align}
        \forall c \in \Sigma: \Lang{D_c(r)} \coloneqq \{ w \in \Sigma^\ast \mid c w \in \Lang{r} \}
    \end{align}
\end{definition}

It is possible to derive regular expressions by using an algorithm which operates on the syntactic level.
The idea is always the same, but there exist multiple variants of the algorithm.
These stem from the fact that care has to be taken to simplify intermediary results to ensure termination.
In the following the variant of the algorithm in~\cite{proof-pearl-regular-expression-equivalence} is presented.

\begin{definition}
    The derivative of a regular expression $r$ over the alphabet $\Sigma$
    can be computed by the following recursive algorithm:

    \begin{align}
        D_c(\emptyset)  &\rightarrow \emptyset \\
        D_c(\epsilon)   &\rightarrow \emptyset \\
        D_c(c)          &\rightarrow \epsilon \\
        D_c(a)          &\rightarrow \emptyset                 & \text{if } a \in \Sigma \land a \neq c \\
        D_c(r^\ast)     &\rightarrow D_c(r) \cdot r^\ast \\
        D_c(r + s)      &\rightarrow D_c(r) + D_c(s) \\
        D_c(r \cdot s)  &\rightarrow D_c(r) + D_c(s)           & \text{if } \epsilon \not\in \Lang{r} \\
        D_c(r \cdot s)  &\rightarrow (D_c(r) \cdot s) + D_c(s) & \text{if } \epsilon \in \Lang{r}
    \end{align}
\end{definition}

Especially the last rule provides potential for infinitely growing terms.
Since it is unpleasant to use algorithms which do not terminate,
these growing terms have to be taken care of.
This is done by replacing $\cdot$ and $+$ by the operators $\oplus$ and $\odot$
in the right-hand sides of the rules.
which are defined as follows:

\begin{definition}
    The operators $\oplus$ and $\odot$ perform simplifications according to axioms of Kleene algebra.
    Additionally, they sort the terms according to some total order $\preceq$ on regular expressions.

    \begin{align}
        \emptyset \odot \_     &\rightarrow \emptyset \\
        \_ \odot \emptyset     &\rightarrow \emptyset \\
        \epsilon \odot r       &\rightarrow r \\
        r \odot \epsilon       &\rightarrow r \\
        (r \cdot s) \odot t    &\rightarrow r \cdot (s \odot t) \\
        \text{else } r \odot s &\rightarrow r \cdot s
    \end{align}

    \begin{align}
        \emptyset \oplus r &\rightarrow r \\
        r \oplus \emptyset &\rightarrow r \\
        (r + s) \oplus t   &\rightarrow r \oplus (s \oplus t) \\
        r \oplus (s + t)   &\rightarrow
        \begin{cases}
            s + t            & \text{if } r = s \\
            r + (s + t)      & \text{if } r \preceq s \\
            s + (r \oplus t) & \text{otherwise}.
        \end{cases}
    \end{align}
\end{definition}

\begin{definition}
    The derivative can be extended to words by an inductive definition:

    \begin{align}
        \label{def:derivative_wrt_words}
        \Lang{D_v(r)} &\coloneqq \{ u \in \Sigma^\ast \mid v u \in \Lang{r} \}
    \end{align}
\end{definition}

\begin{theorem}
    The derivative of a regular expression $r$ w.r.t. a word $w$ can be calculated with the following algorithm:

    \begin{align}
        D_\epsilon(r) &\rightarrow r \\
        \forall a \in \Sigma, w \in \Sigma^\ast: D_{a w}(r) &\rightarrow D_w(D_a(r))
    \end{align}
\end{theorem}

\begin{proof}
    We have to prove that both rules compute the right language.
    The former is trivially shown because it reduces to the same regular expression.

    For the latter, first we use the definition of $\Lang{D_w(r)}$ in~\ref{def:derivative_wrt_words}.
    By the definition of the derivative w.r.t. single characters,
    we know that iff $w~v~\in~\Lang{D_a(r)}$, then $a w v \in \Lang{r}$.
    Therefore:
    \begin{align}
        \Lang{D_w(r)}
        = \{ v \in \Sigma^\ast \mid w v \in \Lang{D_a(r)} \}
        = \{ v \in \Sigma^\ast \mid a w v \in \Lang{r} \}
        = \Lang{D_{a w}(r)}
    \end{align}
\end{proof}

\begin{theorem}
    A word $w$ is part of the language represented by a regular expression $r$ iff
    the language of the derivative of $r$ w.r.t. $w$ contains the empty set:

    \begin{align}
        w \in \Lang{r} \Leftrightarrow \epsilon \in \Lang{D_w(r)}
    \end{align}
\end{theorem}

\begin{proof}
    \begin{align}
        w \in \Lang{r} \Leftrightarrow w \epsilon \in \Lang{r} \Leftrightarrow \epsilon \in \Lang{D_w(r)}
    \end{align}

    The last step follows from~\ref{def:derivative_wrt_words}
\end{proof}
