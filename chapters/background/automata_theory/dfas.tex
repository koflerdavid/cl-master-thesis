Deterministic finite automata (DFAs) can be thought of as graphs with labeled vertices and edges.

\begin{definition}
    A DFA can be defined using the following five components:

    \begin{itemize}
        \item A set of vertices $Q_D$, called states.
        As implied by the term DFA, this set has to be finite.
        \item A set of labels $\Sigma_D$ on the vertices which is called the alphabet of input symbols.
        It has to be finite as well.
        \item An ``initial'' state $s_D \in Q_D$.
        \item A number of states $F_D \subseteq Q_D$ called accepting states.
        \item A set of directed edges called transitions.
        Each transition is labelled with one input symbol.
        Furthermore, for each input symbol there is exactly one transition labelled with
        the symbol between each pair of states.
        Because of this, this set can be represented by a total function
        $\delta_D : Q_D \times \Sigma_N \rightarrow Q_D$ which associates a successor state to
        every pair of state and input symbol.
    \end{itemize}
\end{definition}

DFAs are chiefly used to represent transition systems.
A system modeled by the automaton is thought as always being in one of the states contained in the automata.
When the system receives an input, a transition to another state is possible.

For certain classes of languages, a DFA can be used to decide whether a specific word $w$ is contained in a language $L$.
This is done by first constructing an appropriate automaton $M$ for the language.
Then, the transition system is initialized to be at the initial state $s_{M}$.
Then, the word $w$ is treated as a sequence of input symbols $c_1 c_2 \ldots c_n$.
For each input symbol, the state transition $\delta_{M}(q, c)$ matching the current state $q$ and input symbol $c$ is followed.
If the system reaches one of the accepting states and there are no more input symbols left, then the
automaton is said to accept the input.

The following lines formally describe this procedure:
\begin{align}
    w \in L \Leftrightarrow \hat\delta_{M_L}(s_{M_L}, w) \in F_{M_L}
\end{align}

\begin{definition}
    $\hat\delta$ is the transition function of the automaton $M$, extended to words.
    This is done by using induction over the length of the input word.
    \begin{align}
        \hat\delta_M(q, \epsilon) &= q\\
        \hat\delta_M(q, c w) &= \hat\delta_M(\delta_M(q, c), w)
    \end{align}
\end{definition}

\begin{definition}
    The set of words accepted by starting from a state $q$ is said to be the language of $q$.
    The language accepted by the initial state of an automata $M$ is the language of $M$.

    \begin{align}
        \Lang{q} &= \{ w \in \Sigma^\ast \mid \hat\delta_M(q, w) \in F_M \}\\
        \Lang{M} &= \Lang{s_M}
    \end{align}
\end{definition}
