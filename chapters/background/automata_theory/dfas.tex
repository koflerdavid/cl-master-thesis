There are two major differences between deterministic finite automata (DFAs) and NFAs .
Firstly, there can only be one initial state.
Also, for every state state and input symbol there
has to be exactly one transition to a successor state.
These two properties make DFAs significantly easier to describe and to handle.

The latter requirement is often inconvenient because NFAs can easily define
failure conditions by just specifying the ``happy'' paths.
In practice, this is significant especially for large alphabets, e.g.,
the set of Unicode characters.
Implementations of DFAs can optimize these common cases and
make all missing transitions go to a designated ``failure'' state.

\begin{definition}
    A DFA can be defined using the following five components:

    \begin{itemize}
        \item A set of vertices $Q_D$, called states.
            Also for DFAs this set has to be finite.
        \item A set of labels $\Sigma_D$ on the vertices,
            which is called the alphabet of input symbols.
            It has to be finite as well.
        \item An initial state $s_D \in Q_D$.
        \item A number of vertices $F_D \subseteq Q_N$ called accepting states.
        \item A set of directed edges called transitions.
            Each transition is labelled with one input symbol.
            Furthermore, for each input symbol there is
            exactly one transition labelled with the symbol between each pair of states.
            Because of this, this set can be represented by a total function
            $\delta_D : Q_D \times \Sigma_N \rightarrow Q_D$
            that associates a successor state to every pair of states and input symbols.
    \end{itemize}
\end{definition}

\begin{definition}
    $\hat\delta$ is the transition function of the automaton $M$, extended to words.
    This is done by using induction over the length of the input word.
    \begin{align}
        \hat\delta_M(q, \epsilon) &= q\\
        \hat\delta_M(q, c w) &= \hat\delta_M(\delta_M(q, c), w)
    \end{align}
\end{definition}

Since computing $\hat\Delta$ for every pair of states and input symbols of an NFA
is a deterministic operation,
it is possible to transform every NFA into a DFA accepting the same language.
The core idea is to regard the current set of possible states as the current state of
this equivalent DFA.
The following theorem makes this transformation explicit:

\begin{theorem}
    Every NFA $N$ can be converted into a DFA $D_N$ as follows:
    \begin{align}
        Q_D &= \Powerset{Q_N} \\
        \Sigma_D &= \Sigma_N \\
        s_D &= S_N \\
        f_D &= \{ S \in \Powerset{F_N} \mid S \cap F_N \neq \{\} \} \\
    \end{align}

    In particular
    \begin{align}
        \forall S \subseteq Q_N, w \in \Sigma^\ast: \hat\delta(S, w) = \hat\Delta(S, w)
    \end{align}
\end{theorem}

The disadvantage of this transformation is that
in general the generated DFA has exponentially more states
than the original NFA since $|Q_D| = |\Powerset{Q_N}| = 2^{|Q_N|}$.
In many cases it is possible to reduce this amount by either pruning states
that are not reachable from starting states,
or by not even computing them in the first place.
This is done by exploring the NFA from the starting states.
