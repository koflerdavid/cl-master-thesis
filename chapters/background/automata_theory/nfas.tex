Nondeterministic finite automata (NFAs) are defined almost the same way as DFAs.
There are two major differences: There can be more than one initial state.
Also, multiple transitions carrying the same input symbol can connect a state to different
successor states.

\begin{definition}
    A NFA can be defined using the following five components:

    \begin{itemize}
        \item A set of vertices $Q_N$, called states.
        Also for NFAs this set has to be finite.
        \item A set of labels $\Sigma_N$ on the vertices which is called the alphabet of input symbols.
        It has to be finite as well.
        \item A set of vertices called ``initial'' states $S_N \subseteq Q_N$.
        \item A number of states $F_N \subseteq Q_N$ called accepting states.
        \item A set of directed edges called transitions.
        Each transition is labelled with one input symbol.
        This set can be represented by a total function
        $\Delta_N : Q_N \times \Sigma_N \rightarrow \mathcal{P}(Q_N)$ which
        associates a set of successor states to every pair of states and input symbols.
    \end{itemize}
\end{definition}

Also NFAs can be used to decide whether a word $w$ is decided in a language $L$.
The idea is the same as for DFAs: the word $w$ is treated as a sequence of input symbols.
This sequence is then again used to traverse the graph from a starting state.
If the traversal ends in an accepting state then the word is said to be accepted.

NFAs complicate this procedure though because from every state more than one
transition, or no transition at all, is allowed to lead to successor states.
Also, there can be more than one initial state, or none at all.
Therefore it is in general not possible to know up front
a sequence of transitions which will lead to an accepting state
for a given sequence of input symbols.

A solution is to simulate all possible executions of the automaton.
This is done by extending the transition function to take
a set of states and a word as input.
Before every step there will then be a set of states the automaton
could currently occupy.
This set, $\Delta$ and the input symbol is then used to compute the possible successor states.

\begin{definition}
    The extended transition function $\hat\Delta$ takes a set of states and a
    word as input and recursively computes all states the automaton could occupy
    after traversing the automaton as indicated by the word.

    \begin{align}
        \hat\Delta(R, \epsilon) &= R \\
        \hat\Delta(R, c w) &= \hat\Delta(\bigcup\limits_{r \in R} \Delta(r, c), w)
    \end{align}
\end{definition}

The function $\hat\Delta$ can now be used to determine whether a word is accepted by an NFA:

\begin{align}
    w \in L \Leftrightarrow \hat\Delta(S_{M_L}, w) \subseteq F_{M_L}
\end{align}

SThis transformation to view every NFA as a DFA.
That DFA's states are all possible subsets of the state set of the NFA.

Since computing $\hat\Delta$ is actually a perfectly deterministic operation
we have actually just transformed NFA into a DFA.
The following definition makes this transformation explicit:

\begin{definition}
    Every NFA $N$ can be converted into a DFA $D_N$ as follows:
    \begin{align}
        Q_D &= \Powerset{Q_N} \\
        \Sigma_D &= \Sigma_N \\
        s_D &= S_N \\
        f_D &= \{ S \in \Powerset{F_N} \mid S \cap F_N \neq \{\} \} \\
        \delta_D(S, c) &= \hat\Delta_N(S, c)
    \end{align}
\end{definition}

The disadvantage of this transformation is that
in general the generated DFA has exponentially more states
than the original NFA since $|Q_D| = |\Powerset{Q_N} = 2^{|Q_N|}$.
In many cases it is possible to reduce this amount by either pruning states
which are not reachable from starting states,
or by not even computing them in the first place.
This is done by exploring the NFA from the starting states.
