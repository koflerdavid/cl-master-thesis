Nondeterministic finite automata (NFAs) can be thought of
as graphs with labeled vertices and edges.

\begin{definition}
    A NFA can be defined using the following five components:

    \begin{itemize}
        \item A set of vertices $Q_N$, called states.
            As implied by the term NFA, this set has to be finite.
        \item A set of labels $\Sigma_N$ on the vertices,
            which is called the alphabet of input symbols.
            It has to be finite as well.
        \item A set of vertices called initial states $S_N \subseteq Q_N$.
        \item A set of vertices $F_N \subseteq Q_N$ called accepting states.
        \item A set of directed edges called transitions.
            Each transition is labelled with one input symbol.
            This set can be represented by a total function
            $\Delta_N : Q_N \times \Sigma_N \rightarrow \Powerset{Q_N}$
            that associates a set of successor states to every pair of states and input symbols.
    \end{itemize}
\end{definition}

Finite automata are often used to describe transition systems.
A system modeled by an automaton is thought as
always being in one of the states contained in the automaton.
When the system receives an input, a transition to another state is made.

For certain classes of languages, a NFA can be used to decide
whether a specific word $w$ is contained in a language $L$.
This is done by first constructing an appropriate automaton $M$ for the language.
Then, the transition system is initialized to be at one of the initial states $s \in S_M$.
Then, the word $w$ is treated as a sequence of input symbols $c_1 c_2 \ldots c_n$.
For each input symbol, a state transition is made to a state from $\Delta_{M}(q, c)$,
where $q$ is the current state $c$ is the input symbol.
If the system reaches one of the accepting states and there are no more input symbols left,
then the automaton is said to accept the input.

\begin{definition}
    The set of words accepted by starting from a set of states $P$
    is said to be the language of $P$.
    The language accepted by the set of initial states of an automata $M$ is the language of $M$.

    \begin{align}
        \Lang{P} &= \{ w \in \Sigma^\ast \mid \hat\Delta_M(P, w) \cap F_M \neq \{ \} \}\\
        \Lang{M} &= \Lang{S_M}
    \end{align}
\end{definition}

\begin{definition}
    The family of languages that can be described with a NFA
    accepting their repsective words is called the regular languages.
\end{definition}

In general though, the previous procedure is complicated because from every state more than one
transition, or no transition at all, can lead to successor states.
Also, there can be more than one initial state, or none at all.
Therefore it is in general not possible to know up front
a sequence of transitions that will lead to an accepting state.

A solution is to simulate all possible executions of the automaton, i.e.,
it is assumed that the automaton is at more than one, or no, states at any possible moment.
This is done by extending the transition function to take
a set of states and a word as input.
This set, $\Delta$ and the input symbol is then used to compute the possible successor states.

\begin{definition}
    The extended transition function $\hat\Delta$ takes a set of states and a
    word as input and computes all states the automaton could occupy
    after traversing the automaton as indicated by the word.

    \begin{align}
        \hat\Delta(R, \epsilon) &= R \\
        \hat\Delta(R, c w) &= \hat\Delta(\bigcup\limits_{r \in R} \Delta(r, c), w)
    \end{align}
\end{definition}

The function $\hat\Delta$ can now be used to determine whether
a word is accepted by a set of states $P$:

\begin{align}
    w \in \Lang{P} \iff \hat\Delta(P, w) \subseteq F_{M_L} \\
    w \in L \iff \hat\Delta(S_{M_L}, w) \subseteq F_{M_L}
\end{align}
