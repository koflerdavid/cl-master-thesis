\chapter{Theoretical Background}

\section{Automata Theory}

    Automata theory is concerned with the study of abstract automata and computational models.
    The aim is to study their essential properties and which types of computational problems can be solved by using them.

    \subsection{Basic Definitions}
    \begin{definition}
    The set of all finite sequences of an alphabet $\Sigma$ is written as $\Sigma^\ast$.
    It is formally defined as
    \begin{align}
        \Sigma^\ast = \bigcup\limits_{n=0}^\infty \{c_1 c_2 \ldots c_n \mid c_i \in \Sigma \text{ for all } 1 \le i \le n \}
    \end{align}
    Subsets of $\Sigma^\ast$ are often called languages.
\end{definition}

The empty sequence (containing no symbols at all) is denoted by $\epsilon$.

\begin{definition}
    A \textit{Kleene algebra} is a set $A$ together with:
    \begin{itemize}
        \item Two designated elements $0$ and $1$
        \item Two binary operations $+, \cdot : A \rightarrow A$.
        \item A unary operation $(-)^\ast : A \rightarrow A$, also called \textit{asterate} or \textit{Kleene star}.
    \end{itemize}

    For all $a, b, c \in A$, the following axioms must hold:

    \begin{itemize}
        \item The operations $+$ and $\cdot$ must be associative:
            \begin{align}
                a \cdot (b \cdot c) &= (a \cdot b) \cdot c\\
                a + (b + c) &= (a + b) + c
            \end{align}

        \item $+$ must be commutative, idempotent, and distributive over $\cdot$:
            \begin{align}
                a + b &= b + a\\
                a + a &= a\\
                (a + b) \cdot c &= a \cdot c + b \cdot c
            \end{align}

        \item $0$ and $1$ are neutral elements for $+$ and $\cdot$, respectively;
            $0$ must be the annihilating element of $\cdot$:
            \begin{align}
                a + 0 &= a\\
                a \cdot 1 &= 1 \cdot a = a\\
                a \cdot 0 &= 0 \cdot a = 0
            \end{align}

        \item Given the partial order defined by $a + b = b \Rightarrow a \leq b$,
            the operation $\ast$ must behave as follows:
            \begin{align}
                1 + a \cdot a^\ast = 1 + a^\ast = a^\ast\\
                b + a \cdot c \leq c \Rightarrow a^\ast \cdot b \leq c\\
                b + c \cdot a \leq c \Rightarrow b \cdot a^\ast \leq c
            \end{align}

    \end{itemize}

\end{definition}

\begin{theorem}
    The simplest example are Boolean Algebras with $\vee$ for $+$, $\wedge$ for $\cdot$,
    and with the constant function $\mathbf{1}$ for $\ast$.
\end{theorem}

\begin{theorem}
    Languages form a Kleene algebra with the following definitions:
    \begin{itemize}
        \item The empty set $\emptyset$ for $0$.
        \item The language $\{ \epsilon \}$ for $1$.
        \item Set union for $+$.
        \item Language concatenation for $\times$, which is defined as follows:
            \begin{align}
                L_1 L_2 \coloneqq \{ v w \mid v \in L_1, w \in L_2 \}
            \end{align}
        \item Asterate for languages is defined as follows:
            \begin{align}
                L^0 &\coloneqq \{\epsilon\} \\
                L^n &\coloneqq L L^{n-1} \\
                L^\ast &\coloneqq \bigcup\limits_{i = 0}^\infty L^i,
            \end{align}
    \end{itemize}
\end{theorem}


    \subsection{Nondeterministic Finite Automata}
    Nondeterministic finite automata (NFAs) are defined almost the same way as DFAs.
There are two major differences: There can be more than one initial state.
Also, multiple transitions carrying the same input symbol can connect a state to different
successor states.

\begin{definition}
    A NFA can be defined using the following five components:

    \begin{itemize}
        \item A set of vertices $Q_N$, called states.
        Also for NFAs this set has to be finite.
        \item A set of labels $\Sigma_N$ on the vertices,
        which is called the alphabet of input symbols.
        It has to be finite as well.
        \item A set of vertices called ``initial'' states $S_N \subseteq Q_N$.
        \item A number of states $F_N \subseteq Q_N$ called accepting states.
        \item A set of directed edges called transitions.
        Each transition is labelled with one input symbol.
        This set can be represented by a total function
        $\Delta_N : Q_N \times \Sigma_N \rightarrow \mathcal{P}(Q_N)$
        that associates a set of successor states to every pair of states and input symbols.
    \end{itemize}
\end{definition}

Also NFAs can be used to decide whether a word $w$ is decided in a language $L$.
The idea is the same as for DFAs: the word $w$ is treated as a sequence of input symbols.
This sequence is then again used to traverse the graph from a starting state.
If the traversal ends in an accepting state then the word is said to be accepted.

NFAs complicate this procedure though because from every state more than one
transition, or no transition at all, is allowed to lead to successor states.
Also, there can be more than one initial state, or none at all.
Therefore it is in general not possible to know up front
a sequence of transitions that will lead to an accepting state
for a given sequence of input symbols.

A solution is to simulate all possible executions of the automaton.
This is done by extending the transition function to take
a set of states and a word as input.
Before every step there will then be a set of states the automaton
could currently occupy.
This set, $\Delta$ and the input symbol is then used to compute the possible successor states.

\begin{definition}
    The extended transition function $\hat\Delta$ takes a set of states and a
    word as input and recursively computes all states the automaton could occupy
    after traversing the automaton as indicated by the word.

    \begin{align}
        \hat\Delta(R, \epsilon) &= R \\
        \hat\Delta(R, c w) &= \hat\Delta(\bigcup\limits_{r \in R} \Delta(r, c), w)
    \end{align}
\end{definition}

The function $\hat\Delta$ can now be used to determine whether a word is accepted by an NFA:

\begin{align}
    w \in L \Leftrightarrow \hat\Delta(S_{M_L}, w) \subseteq F_{M_L}
\end{align}

SThis transformation to view every NFA as a DFA.
That DFA's states are all possible subsets of the state set of the NFA.

Since computing $\hat\Delta$ is actually a perfectly deterministic operation
we have actually just transformed NFA into a DFA.
The following definition makes this transformation explicit:

\begin{definition}
    Every NFA $N$ can be converted into a DFA $D_N$ as follows:
    \begin{align}
        Q_D &= \Powerset{Q_N} \\
        \Sigma_D &= \Sigma_N \\
        s_D &= S_N \\
        f_D &= \{ S \in \Powerset{F_N} \mid S \cap F_N \neq \{\} \} \\
        \delta_D(S, c) &= \hat\Delta_N(S, c)
    \end{align}
\end{definition}

The disadvantage of this transformation is that
in general the generated DFA has exponentially more states
than the original NFA since $|Q_D| = |\Powerset{Q_N} = 2^{|Q_N|}$.
In many cases it is possible to reduce this amount by either pruning states
that are not reachable from starting states,
or by not even computing them in the first place.
This is done by exploring the NFA from the starting states.


    \subsection{Deterministic Finite Automata}
    Deterministic finite automata (DFAs) can be thought of as graphs with labeled vertices and edges.

\begin{definition}
    A DFA can be defined using the following five components:

    \begin{itemize}
        \item A set of vertices $Q_D$, called states.
            As implied by the term DFA, this set has to be finite.
        \item A set of labels $\Sigma_D$ on the vertices,
            which is called the alphabet of input symbols.
            It has to be finite as well.
        \item An ``initial'' state $s_D \in Q_D$.
        \item A number of states $F_D \subseteq Q_D$ called accepting states.
        \item A set of directed edges called transitions.
        Each transition is labelled with one input symbol.
        Furthermore, for each input symbol there is
        exactly one transition labelled with the symbol between each pair of states.
        Because of this, this set can be represented by a total function
        $\delta_D : Q_D \times \Sigma_N \rightarrow Q_D$
        that associates a successor state to every pair of state and input symbol.
    \end{itemize}
\end{definition}

DFAs are often used to describe transition systems.
A system modeled by an automaton is thought as
always being in one of the states contained in the automaton.
When the system receives an input, a transition to another state is made.

For certain classes of languages, a DFA can be used to decide
whether a specific word $w$ is contained in a language $L$.
This is done by first constructing an appropriate automaton $M$ for the language.
Then, the transition system is initialized to be at the initial state $s_{M}$.
Then, the word $w$ is treated as a sequence of input symbols $c_1 c_2 \ldots c_n$.
For each input symbol, the state transition $\delta_{M}(q, c)$
matching the current state $q$ and input symbol $c$ is followed.
If the system reaches one of the accepting states and there are no more input symbols left,
then the automaton is said to accept the input.

The following lines formally describe this procedure:
\begin{align}
    w \in L \Leftrightarrow \hat\delta_{M_L}(s_{M_L}, w) \in F_{M_L}
\end{align}

\begin{definition}
    $\hat\delta$ is the transition function of the automaton $M$, extended to words.
    This is done by using induction over the length of the input word.
    \begin{align}
        \hat\delta_M(q, \epsilon) &= q\\
        \hat\delta_M(q, c w) &= \hat\delta_M(\delta_M(q, c), w)
    \end{align}
\end{definition}

\begin{definition}
    The set of words accepted by starting from a state $q$ is said to be the language of $q$.
    The language accepted by the initial state of an automata $M$ is the language of $M$.

    \begin{align}
        \Lang{q} &= \{ w \in \Sigma^\ast \mid \hat\delta_M(q, w) \in F_M \}\\
        \Lang{M} &= \Lang{s_M}
    \end{align}
\end{definition}

\begin{definition}
    The family of languages that can be described with a DFA
    accepting its words is called the regular languages.
\end{definition}


    \subsection{Regular Expressions}
    Regular expressions are a formal language that is used to represent languages.
Regular expressions are equivalent in expressive power to the DFAs and NFAs presented above,
but in many cases humans find them more pleasant to work with.

\begin{definition}
    The language $R_\Sigma$, the set of regular expressions over the alphabet $\Sigma$,
    is inductively defined as the smallest set satisfying the following rules:

    \begin{itemize}
        \item The empty set is a regular expression:
        $\emptyset \in R_\Sigma$.
        \item The empty string $\epsilon$ is a regular expression:
        $\epsilon \in R_\Sigma$.
        \item A character from $\Sigma$, like $a$, is a regular expression:
        $\forall c \in \Sigma: c \in R_\Sigma$.
        \item Two regular expressions, concatenated using the operator $\cdot$, is a regular expression:
        $\forall r, s \in R_\Sigma: r \cdot s \in R_\Sigma$.
        In most cases, the operator is omitted to make the regular expression more concise.
        \item Two regular expressions, combined using the operator $+$, form a regular expression:
        $\forall r, s \in R_\Sigma: r + s \in R_\Sigma$.
        \item An application of the postfix operator $\ast$ to a regular expression is a regular expression:
        $\forall r \in R_\Sigma: r^\ast \in R_\Sigma$.
        \item A regular expression, surrounded by $()$, is a regular expression:
        $\forall r \in R_\Sigma: (r) \in R_\Sigma$.
    \end{itemize}
\end{definition}

As with every formal language, a semantics is required for regular expressions to define their meaning.

\begin{definition}
    The following inductive definitions map regular expressions over an alphabet $\Sigma$ to languages:

    \begin{itemize}
        \item $\Lang{\emptyset} = \{\}$
        \item $\forall c \in \Sigma: \Lang{c} \coloneqq \{c\} $
        \item $\forall r, s \in R_\Sigma: \Lang{r \cdot s} \coloneqq \Lang{r} \Lang{s}$
        \item $\forall r, s \in R_\Sigma: \Lang{r + s} \coloneqq \Lang{r} \cup \Lang{s}$
        \item $\forall r \in R_\Sigma: \Lang{r^\ast} \coloneqq \Lang{r}^\ast$
    \end{itemize}
\end{definition}


    \subsection{Derivatives of Regular Expressions}
    Taking the Brzozowski derivative of regular expressions is an interesting operation which, among others, can be used
to implement regular expressions in software, and also to convert regular expressions into finite automata.
It can also be extended to other classes of formal languages~\cite{parsing-with-derivatives}.
The operation has similarities to taking the (partial) derivative of arithmetic expressions.

It is possible to derive regular expressions by using an algorithm which operates on the syntactic level.
The idea is always the same, but there exist multiple variants of the algorithm.
These stem from the fact that care has to be taken to simplify intermediary results to ensure termination.
In the following the variant of the algorithm in~\cite{proof-pearl-regular-expression-equivalence} is presented.

\begin{definition}
    The derivative of a regular expression $r$ over the alphabet $\Sigma$
    can be computed by the following recursive algorithm:

    \begin{align}
        D_c(\emptyset)  &\rightarrow \emptyset \\
        D_c(\epsilon)   &\rightarrow \emptyset \\
        D_c(c)          &\rightarrow \epsilon \\
        D_c(a)          &\rightarrow \emptyset                 & \text{if } a \in \Sigma \land a \neq c \\
        D_c(r^\ast)     &\rightarrow D_c(r) \cdot r^\ast \\
        D_c(r + s)      &\rightarrow D_c(r) + D_c(s) \\
        D_c(r \cdot s)  &\rightarrow D_c(r) + D_c(s)           & \text{if } \epsilon \not\in \Lang{r} \\
        D_c(r \cdot s)  &\rightarrow (D_c(r) \cdot s) + D_c(s) & \text{if } \epsilon \in \Lang{r}
    \end{align}
\end{definition}

Especially the last rule provides potential for infinitely growing terms.
Since it is unpleasant to use algorithms which do not terminate,
these growing terms have to be taken care of.
This is done by replacing $+$ and $\cdot$ with the operators $\oplus$ and $\odot$
in the right-hand sides of the rules.
which are defined as follows:

\begin{definition}
    The operators $\oplus$ and $\odot$ perform simplifications according to axioms of Kleene algebra.
    Additionally, they sort the terms according to some total order $\preceq$ on regular expressions.

    \begin{align}
        \emptyset \odot \_     &\rightarrow \emptyset \\
        \_ \odot \emptyset     &\rightarrow \emptyset \\
        \epsilon \odot r       &\rightarrow r \\
        r \odot \epsilon       &\rightarrow r \\
        (r \cdot s) \odot t    &\rightarrow r \cdot (s \odot t) \\
        \text{else } r \odot s &\rightarrow r \cdot s
    \end{align}

    \begin{align}
        \emptyset \oplus r &\rightarrow r \\
        r \oplus \emptyset &\rightarrow r \\
        (r + s) \oplus t   &\rightarrow r \oplus (s \oplus t) \\
        r \oplus (s + t)   &\rightarrow
        \begin{cases}
            s + t            & \text{if } r = s \\
            r + (s + t)      & \text{if } r \preceq s \\
            s + (r \oplus t) & \text{otherwise}.
        \end{cases}
    \end{align}
\end{definition}

\begin{theorem}
    The derivative of a regular expression $r$ w.r.t. a letter $c$ is
    the set of all words in $\Lang{r}$ which start with $c$:

    \begin{align}
        \forall c \in \Sigma: \Lang{D_c(r)} \coloneqq \{ w \in \Sigma^\ast \mid c w \in \Lang{r} \}
    \end{align}
\end{theorem}

\begin{definition}
    The derivative of a regular expression $r$ w.r.t. a word $w$ can be calculated with the following algorithm:

    \begin{align}
        D_\epsilon(r) &\rightarrow r \\
        \forall a \in \Sigma, w \in \Sigma^\ast: D_{a w}(r) &\rightarrow D_w(D_a(r))
    \end{align}
\end{definition}

\begin{definition}
    The derivative is extended to words by the following definition:

    \begin{align}
        \label{def:derivative_wrt_words}
        \Lang{D_v(r)} &\coloneqq \{ u \in \Sigma^\ast \mid v u \in \Lang{r} \}
    \end{align}
\end{definition}

\begin{proof}
    We have to prove that both rules compute the right language.
    The former is trivially shown because it reduces to the same regular expression.

    For the latter, first we use the definition of $\Lang{D_w(r)}$ in~\ref{def:derivative_wrt_words}.
    By the definition of the derivative w.r.t. single characters,
    we know that iff $w~v~\in~\Lang{D_a(r)}$, then $a w v \in \Lang{r}$.
    Therefore:
    \begin{align}
        \Lang{D_w(r)}
        = \{ v \in \Sigma^\ast \mid w v \in \Lang{D_a(r)} \}
        = \{ v \in \Sigma^\ast \mid a w v \in \Lang{r} \}
        = \Lang{D_{a w}(r)}
    \end{align}
\end{proof}

\begin{theorem}
    A word $w$ is part of the language represented by a regular expression $r$ iff
    the language of the derivative of $r$ w.r.t. $w$ contains the empty set:

    \begin{align}
        w \in \Lang{r} \Leftrightarrow \epsilon \in \Lang{D_w(r)}
    \end{align}
\end{theorem}

\begin{proof}
    \begin{align}
        w \in \Lang{r} \Leftrightarrow w \epsilon \in \Lang{r} \Leftrightarrow \epsilon \in \Lang{D_w(r)}
    \end{align}

    The last step follows from~\ref{def:derivative_wrt_words}
\end{proof}


    \subsection{Equivalence of Finite Automata and Regular Expressions}

\section{Automata Equivalence and Inclusion}

    One of the most interesting applications of automata theory are model~checking and verification.
    The goal is to check whether a specification has certain properties
    or whether the realisation of a specification behaves as expected.
    Naturally, the question of equivalence between two model arises:
    how can two automata, maybe a simulated and a real one reconstructed from a trace,
    shown to be equivalent?

    The following section explains the necessary theoretical background and presents a
    new development in this field.

    \subsection{Automata Equivalence}
    The following definitions transform the question of
whether two automata are equivalent to
whether the set of their starting states are equivalent.
This definition is more general and useful
for the exposition of the algorithms that determine equivalence.

\begin{definition}
    Two finite automata $M$ and $N$ are merged with the following definitions.
    \begin{align}
        Q_{M \sqcup N} = Q_M \cup Q_N \\
        \Sigma_{M \sqcup N} = \Sigma_M \cup \Sigma_N
    \end{align}

    The sets $Q_M$ and $Q_N$ have to be disjoint, which
    can always be ensured by renaming states.
    This is required to ensure that the languages of each subset of states are preserved.
\end{definition}

\begin{definition}
    Two sets of states $P$ and $Q$ of an NFA $M$ are considered
    to be equivalent iff they accept the same language:

    \begin{align}
        P \equiv Q \iff \Lang{P} = \Lang{Q}
    \end{align}
\end{definition}


    \subsection{Coalgebras and Bisimulations}
    \begin{definition}
    A category is a class of \textit{objects}, together with \textit{morphisms} between them.
    A morphism maps an \textit{origin} object to a \textit{destination} object.
    For every object, we require the existence of an \textit{identity} morphism $id$ from the object to itself.

    Also, we require composability between morphisms:
    Two morphisms can be \textit{composed} if the source object of the first is
    the destination object of the second.
    For two morphisms $f : B \rightarrow C$ and $g : A \rightarrow B$,
    this is described as $f \circ g : A \rightarrow C$.

    The identity morphism must behave as a neutral element regarding composition, i.e.,
    if $f : A \rightarrow B$ then $id_B \circ f = f \circ id_A = f$.
\end{definition}

\begin{example}
    The class of all possible sets as objects, together with functions between them as morphisms,
    yields the category \textbf{Set}.
    The identity function $id_A$ acts as the identity morphism for every set $A$.
    Clearly, function composition satisfies the above requirements.
\end{example}

\begin{example}
    Every graph $G$ with the set $V$ as vertices and the set $E$ as edges
    can be turned into a category with the set $V$ as objects
    using the \textit{Free Category} construction.
    To fulfil the category laws, its morphisms are drawn from
    the \textit{paths} inside the graph:

    \begin{itemize}
        \item There is a path from every vertex $v \in V$ to itself.
        \item For all vertices $u, v, w \in V$,
              every path $p$ from $u$ to $v$,
              and all edges $e \in E$ from $v$ to $w$,
              there is a path $q$ from $u$ to $w$.
    \end{itemize}
\end{example}

\begin{definition}
    The morphisms of the category of \textit{small categories} are called \textit{functors}.
    A \textit{functor} $F : A \rightarrow B$ maps objects and mappings between categories.
    Applying a functor to a mapping is also called \textit{lifting}.
    Functors have to preserve the following properties:

    \begin{itemize}
        \item For every morphism $f$ and $g$ from $A$, $F (g \circ f)$
              must be equivalent to $F g \circ F f$.
        \item The identity morphism of an object $a$ from $A$, $F(id_{a}) $,
              must be lifted to $id_{F(a)}$, the identity morphism of $F(a)$.
    \end{itemize}

    Functors that map a category to itself are called \textit{endofunctors}.
\end{definition}

\begin{definition}
    Given a category $\Cat{C}$ and an endofunctor $F : \Cat{C} \rightarrow \Cat{C}$,
    every object $A$ from $Cat{C}$, together with a morphism
    $\alpha : F(\Cat{C}) \rightarrow \Cat{C}$
    specifies an $F$-algebra with the object $A$ as \textit{carrier}.
\end{definition}

Algebras are mathematical objects that are equipped with some \textit{constructors}.
A constructor is a function that makes it possible to generate elements of
the carrier algebra\footnote{In the abstract definition, these are combined into a single
constructor by using the disjoint union of functions}, often from existing ones.

\begin{definition}
    Given two functions $a : F \rightarrow G$ and $b : F' \rightarrow G$,
    $[a, b] : (F \sqcup F') \rightarrow G$ maps

\end{definition}

\begin{example}
    In the category \textbf{Set}, the function
    $[\textit{nil}, \textit{cons}] : \textbf{1} \sqcup X \times List(X) \rightarrow List(X)$
    with the carrier $List(X)$ is the algebra of lists of elements of the set $X$.
    This algebra has one constructor that maps the singleton set to the empty list,
    or prepends an element of $X$ to another list.
\end{example}

Coalgebras are mathematical structures which are not defined in terms of constructors
out of which objects are composed, but by destructors by which an object is "taken apart".
They are especially suited for infinite or circular data structures,
such as streams, graphs, automata, processes, etc.
Said differently, algebras \textit{construct} objects
while coalgebras \textit{observe} objects~\cite{Jacobs97atutorial}.

\begin{definition}
    Given a category $\Cat{C}$ and an endofunctor $F : \Cat{C} \rightarrow \Cat{C}$,
    every object $X$ from $\Cat{C}$, together with a morphism
    $\alpha : \mathcal{C} \rightarrow F(\mathcal{C})$
    specifies an $F$-coalgebra with the object $X$ as \textit{carrier}.
\end{definition}

\begin{example}
    Streams are an infinite data structure and therefore
    a prime example for a coinductive data structure.
    The function
    $\textit{next} : \textbf{Stream}(A) \rightarrow A \times \textbf{Stream}(A)$
    is a destructor that decomposes a stream of elements from $A$ into a head and a tail.
    The destructor \textit{next} can be used to define the functions \textit{head} and \textit{tail}.
    This way, a coalgebra might be viewed as having more than one destructor.
\end{example}

\begin{example}
    States of an NFA $M$ can be represented as a coinductive algebra as well.
    To make presentation easier, the destructor has been split into two functions:

    \begin{align}
        o: Q_M \rightarrow \{0, 1\}\quad &o(q) =
            \begin{cases}
                0 & \textbf{if } q \not\in F_M\\
                1 & \textbf{if } q \in F_M
            \end{cases}\\
        t: Q_M \rightarrow {\Powerset{Q_N}} ^{\Sigma_M} \quad & t(q)(a) = \delta_M(q, a)
    \end{align}

    Terms like $t(q)(a)$ are also often written as $t_a(q)$.
    From now on this coalgebraic formulation will be used, i.e.,
    instead of checking whether a state $q$ is an accepting state,
    we will check whether its output $o(q)$ is equal to $1$.
\end{example}

The usage of destructors instead of constructors requires the use of different proof principles.
Bisimulations are such a coinductive proof principle that is used to establish
a notion of \textit{observational equivalence}.
It compares the behavior of elements of the coinductive algebra,
i.e., if the destructors yield the same results for both elements, then these elements are concluded
to be observationally equivalent.
In the following, the application of this proof principle to automata is demonstrated.

\begin{definition}
    A relation $R$ between the states of the DFA $M$ is a bisimulation
    iff $R$ satisfies the following properties:

    \begin{align}
        \forall (p, q) \in R&: o(p) = o(q) \\
        \forall (p, q) \in R, a \in \Sigma_M&: (t_a(p), t_a(q)) \in R
    \end{align}

    In the following, $(p, q) \in R$ will be written more succinctly as $p\;R\;q$.
\end{definition}

\begin{definition}
    The greatest bisimulation over the states of an automaton is the bisimilarity relation $\sim$.
\end{definition}

%\begin{theorem}
%    The bisimilarity relation $\sim$ is an equivalence relation.
%\end{theorem}
%
%\begin{proof}
%    We have to prove the three properties of an equivalence relation:
%    reflexivity, symmetry, and transitivity.
%
%    Reflexivity is easy since for all $s \in Q: o(s) = o(s)$
%    Symmetry is also given because for any $p, q \in R$the statement of the second clause of the
%    definition of bisimulation has to be true for
%    Transitivity
%\end{proof}

\begin{definition}
    A relation $R$ between the states of the DFA $M$ is a bisimulation up to $f$
    if $f$ is a monotone function and if $R$ satisfies the following properties:

    \begin{align}
        \forall (p, q) \in R&: o(p) = o(q) \\
        \forall (p, q) \in R, a \in \Sigma_M&: (t_a(p), t_a(q)) \in f(R)
    \end{align}

    Monotonicity of $f$ is defined as follows:
    \begin{align}
        \forall S \in \Powerset{Q \times Q}: S \subseteq f(S)
    \end{align}
\end{definition}

The properties of $f$ ensure that the resulting bisimulation
is equal to or larger than the original one.
This is very useful to speed up proof algorithms that work by computing a bisimulation:
each element of a bisimulation up-to does not only stand for itself,
but can be used to prove the existence of other elements.
Thus, the computational effort to prove the existence of a certain element
is reduced in the best case.

The following subsection presents an algorithm that can take advantage of bisimulations up-to.


    \subsection{Hopcroft and Karp's Algorithm}
    Hopcroft and Karp's algorithm takes as input a DFA and a pair of states
(usually the starting states) and tries to compute a bisimulation
relating them.
If it succeeds then the two states are proven to accept the same language.
If not, then the algorithm terminates and can be made to return a pair
of states that do not accept the same language.

\begin{definition}
    \begin{align*}
        \underline{\mathit{Naive(P_1, P_2)}}: &\quad (Q \times Q) \to \{\mathbf{true}, \mathbf{false}\} \\
        \text{(1) } & R := \emptyset, \mathit{todo} := \{(P_1, P_2)\} \\
        \text{(2) } & \text{while } \mathit{todo} \neq \emptyset \text{ do}\\
        & \text{(2.1)}\quad \mathit{todo} := \mathit{todo} \setminus \{(P', Q')\}\\
        & \textbf{(2.2)}\quad \textbf{if } \mathbf{(P', Q') \in} \textbf{ R} \textbf{ then goto (2)}\\
        & \text{(2.3)}\quad \text{if } \text{acc}_N(P') \nLeftrightarrow \text{acc}_N(Q') \text{ then return } \mathbf{false}\\
        & \text{(2.4)}\quad \mathit{todo} := \mathit{todo} \cup \bigcup_{a \in \Sigma}{(\delta(P', a), \delta(Q', a))}\\
        & \text{(2.5)}\quad R := R \cup \{(P', Q')\} \\
        \text{(3) } & \text{return } \mathbf{true}\\
    \end{align*}
\end{definition}


    \subsection{Exploitation of Closure Properties}
    The previous algorithm builds up a bisimulation and checks in every step whether
the current constraint is contained in the bisimulation.
As mentioned before, a bisimulation up-to $f$ can speed up the algorithm
because thanks to the function $f$ the bisimulation yields more information.
In the following, the requirements and some examples for this function $f$ are demonstrated.

\begin{definition}
    A relation $R$ \textit{progresses} to a relation $R'$
    (both being relations between states of an automaton $M$)
    if for every pair $(p, q)$ in $R$, the following conditions are valid:

    \begin{itemize}
        \item $o(p) = o(q)$ and
        \item $\forall a \in \Sigma_M: t_a(p) R t_{q)}$
    \end{itemize}

    The symbol $\rightarrowtail$ is used to denote progression.
\end{definition}

\begin{example}
    Any relation $R$ for which $R \rightarrowtail R$ is a bisimulation.
\end{example}

\begin{definition}
    A function $f$ over relations of states is compatible if it is monotone and if
    preserves progressions, i.e., $R \rightarrowtail R' \Rightarrow f(R) \rightarrowtail f(R')$.
\end{definition}

A simple optimization to the algorithm from the previous subsection is
to check membership in a bisimulation up-to equivalence.
This can be efficiently done by using a union-find data structure, which
is actually the algorithm originally devised by Hopcroft and Karp.
This illustrates the power of the bisimulation up-to approach because
using a carefully designed data structure that allows for fast membership check
an algorithm can avoid a lot of redundant computations.

\todo{show that the equivalence closure is compatible}

To use the above algorithm with NFAs, the determinised NFA can be used.
At first sight this makes matters complicated because now sets of states
have to be compared, but it opens up another possibility of optimization
that is based on a simple observation:
for every pair of sets of states $X$ and $Y$,
the language of the union of $X$ and $Y$ is equal to
the union of the languages of $X$ and $Y$.
The main advantage of this observation is that it is not necessary anymore
to visit large parts of the determinised NFA.



    \subsection{Antichain Method}
    \todo{Explain antichain methods in this framework}
